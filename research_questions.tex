\documentclass{article}
\usepackage[utf8]{inputenc}
\usepackage{geometry}
 \geometry{
 a4paper,
 total={170mm,257mm},
 left=20mm,
 top=20mm,
 }
 \usepackage{graphicx}
 \usepackage{titling}

 \title{Forschungsfrage für Sea-Watch Spiel unter Unity
}
\author{Mengfan Wu}
\date{November 2022}
 
 \usepackage{fancyhdr}
\fancypagestyle{plain}{%  the preset of fancyhdr 
    \fancyhf{} % clear all header and footer fields
    \fancyfoot[R]{\includegraphics[width=2cm]{}}
    \fancyfoot[L]{\thedate}
    \fancyhead[L]{}
    \fancyhead[R]{\theauthor}
}
\makeatletter
\def\@maketitle{%
  \newpage
  \null
  \vskip 1em%
  \begin{center}%
  \let \footnote \thanks
    {\LARGE \@title \par}%
    \vskip 1em%
    %{\large \@date}%
  \end{center}%
  \par
  \vskip 1em}
\makeatother

\usepackage{lipsum}  
\usepackage{cmbright}

\begin{document}

\maketitle

\noindent\begin{tabular}{@{}ll}
    Student &  Simon Thomas Dallner \\& Lukas Kifinger \\&  Xun Ma \\& Mengfan Wu
     
\end{tabular}

\section{Kriterien}

\subsection{begrenzt}
Grenzt das Thema Unity ein
\subsection{relevant}
Passend zu unseren Projektinhalten

\subsection{komplex}
eine ganze Arbeit für die Beantwortung nötig

\subsection{in einem Satz}
Fasst das gesamte Projekt zusammen und drückt den Zweck des Projekts aus

\subsection{offen gestellt}
Ausführliche Antworten sind erforderlich,“ja" "nein"  können nicht verwendet werden


\section{Forschungsfrage}
\subsection{Wie entwickelt man ein interaktives Spiel unter Unity?}

\subsection{Wie entwickelt man ein interaktives Spiel „Sea Watch“ basierend auf IT-Lernen und Unity3D}

\subsection{Was sind die Schwerpunkte von dem interaktiven Spiel, die Probleme und Lösung während der Entwirklung? }

\subsection{Wie implementiert man eine IT-basierte Lern-App unter Unity}
\subsection{Wie verwendet man Unity zum Entwerfen und Implementieren von IT-Basiertes Lernen Spiel}

\end{document}
